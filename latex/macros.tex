% Colored boxes like "Go further" or whatever
% The actual boxes are drawn by tcolorbox
\usepackage{tcolorbox}

% First we define a little hash table library by storing things as
% global variables (ugh, but standard)
\newcommand{\defprop}[3]{
  \expandafter\def\csname @val@#1@#2\endcsname{#3}
}

\newcommand{\getprop}[2]{%
  \ifcsname @val@#1@#2\endcsname%
  \csname @val@#1@#2\endcsname%
  \fi
}

% Next, here, we define the different types of boxes
% The colors here are chosen to match the look of the website
\newcommand{\defbox}[3]{
  \defprop{#1}{title}{#2}
  \defprop{#1}{color}{#3}
}

\defbox{quirk}{Quirk}{RoyalPurple}
\defbox{warning}{Warning}{red}
\defbox{installation}{Installation}{blue}
\defbox{further}{Go Further}{ForestGreen}
\defbox{example}{Example}{Gray}
\defbox{notcode}{}{Black}
\defbox{author-picture}{}{Black}

% Finally we define the actual "bookblock" environment that our
% Markdown will expand divs into
\newenvironment{bookblock}[1]{
  \ifcsname @val@#1@color\endcsname
  \begin{tcolorbox}[colback=\getprop{#1}{color}!5!white,
                    colframe=\getprop{#1}{color}!75!white,
                    title={\getprop{#1}{title}}]
  \else
  \PackageError{bookblock}{Unknown book block '#1'}{}
  \fi
}{
\end{tcolorbox}
}

% Same but with no title
\newenvironment{bookblock*}[1]{
  \ifcsname @val@#1@color\endcsname
  \begin{tcolorbox}[colback=\getprop{#1}{color}!5!white,
                    colframe=\getprop{#1}{color}!75!white]
  \else
  \PackageError{bookblock}{Unknown book block '#1'}{}
  \fi
}{
\end{tcolorbox}
}

